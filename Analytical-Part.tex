\chapter{Analytical part}
\label{chap:techniques}

% Project management in the past


\section{E-learning}

Education is one of the most important tools to be successful in life. The knowledge gained through education gives opportunity to achieve better prospects in career growth. Education has played outstanding role in the modern world of Information Technology. And different methods and techniques of learning are being introduced. E-Learning is one of the area that is growing rapidly. And people are arguing about the replacement of traditional classroom learning with e-Learning.
E-learning can be defined as technology-based learning in which remote learners get learning materials electronically via a computer network. By the statistics, in 2011, it was estimated that about \$35.6 billion was spent on self-paced eLearning across the globe. Today, e-Learning is a \$56.2 billion industry, and it's going to double by 2015 \cite{Pappas}. E-Learning is cost-effective for learners, rapid delivery of content is provided, included student-progress monitoring, time and location flexible, potentially available to global audience, learner-centered and self-paced. There are a lot of websites where you can take online courses and get even certificates at the end. 

There was made an experiment. There were created to experimental groups, one of which will study in classrooms and another one will be taught online via Internet. The same instructors who taught the classroom group also prepared online materials for e-learning groups. At the end of experiment the results showed that students can perform better in e-Learning than classroom learning. \cite{Dansong}

Summing up, I would like to call your attention to the fact that e-learning has a big potential to replace the traditional classroom learning. However, it is still in an early stage. And it is mostly depends on learners, on their learning styles.
Internet and multimedia technologies are reshaping the way knowledge is delivered, and that e-learning is becoming a real alternative to traditional classroom learning. Each company wants their employees to be equipped with latest information and skills. To fill this needs there were opened thousands of online courses with programs and certificates. The first non commercial online course was exhibited by MIT university which was freely available through internet. The advantages of e-learning are time and location flexibility, cost effective. The advantage of classroom learning in front of e-learning is immediate feedback and motivated students.

Not all e-learning systems are good. At first time the online courses were boring because of containing only texts materials or unstructured presenting of materials. Nowadays there are used multimedia, audio, pictures and etc. and they are goodly organized. 

The VM(Virtual Mentor), a multimedia-based e-learning environment that enables well-structured, synchronized, and interactive multimedia instructions. The VM consists of following principles: mulitmedia-integration, just-in-time knowledge acquisition, interactively, self-directivity, flexibility and intelligence. There were developed LBA(learning by asking) system were all the materials are placed on the Internet in comprehensive manner. LBA consists of two major parts like Asking-A-Question and Interactive E-Classroom. During the lesson learner can ask a question by either a keywords or conversational English. After the question been asked it sends to the server which immediately shows it to the learners in the virtual room. 

In comparing with e-learning in traditional classrooms learners are capable simultaneously observe through power point presentation and listen for instructor. In LBA has an e-classroom which simulates it. There people can also look through presentation, listen for instructor and see lecture notes in a single web interface. E-classroom automatically shows the corresponding slides and lecture notes about the topic the instructor is introducing in the video. 
Interactive E-Classroom provides rich contents interaction. There is possible to go to the next lecture or to the previous one depending on the learners understanding of the topic. There is also integrated text-based online discussion forum integrated into LBA which allows to exchange messages post comments on posts. All comments are grouped by their topic.

To evaluate e-learning there were been conducted experiment. In the experiment students were randomly selected into one of the groups. The students of first group were been able to sit in tradition classrooms and the students of second group were been able to attend in E-Classroom. The result were analyzed  through the pre-lecture and post-lecture tests. Learning satisfactory was measured by points between 1-7 extremely dissatisfied and extremely satisfied correspondingly. The results of test of students who attended E-Classroom were higher. The satisfaction level of students were approximately same.\cite{Dansong}

Our experiment have demonstrated that e-learning can be at least as effective as traditional classroom learning under certain situations. But e-learning requires more maturity and self-discipline which is not easy for every student. In many cases e-learning can significantly complement classroom learning.

A whiteboard provides a large shared space for the participants to focus their attention and express their ideas, and is therefore a great collaboration tool for information workers. However, it has several limitations; notably, the contents on the whiteboard are hard to archive or share with people who are not present in the discussions. This paper presents our work in developing tools to facilitate collaboration on physical whiteboards by using a camera and a microphone. In particular, we have developed two systems: a whiteboard-camera system and a projector-whiteboardcamera system. The whiteboard-camera system allows a user to take notes of a whiteboard meeting when he/she wants or in an automatic way, to transmit the whiteboard content to remote participants in real time and in an efficient way, and to archive the whole whiteboard session for efficient post-viewing. The projector-whitebord-camera system incorporates a whiteboard into a projectorcamera system. The whiteboard serves as the writing surface (input) as well as the projecting surface (output). Many applications of such a system inevitably require extracting handwritings from video images that contain both handwritings and the projected content. By analogy with echo cancellation in audio conferencing, we call this problem visual echo cancellation, and we describe one approach to accomplish the task. Our systems can be retrofit to any existing whiteboard. With the help of a camera and a microphone and optionally a projector, we are effectively bridging the physical and digital worlds.


\section{Goals and objectives}

The work presented in this paper focuses on the particular meeting scenarios that use whiteboard heavily such as brainstorming sessions, lectures, project planning meetings, and patent disclosures. In those sessions, a whiteboard is indispensable. It provides a large shared space for the participants to focus their attention and express their ideas spontaneously. It is not only effective but also economical and easy to use – all you need is a flat board and several dry-ink pens. While whiteboard sessions are frequent for knowledge workers, they are not perfect. The content on the board is hard to archive or share with others who are not present in the session. People are often busy copying the whiteboard content to their notepads when they should spend time sharing and absorbing ideas. Sometimes they put “Do Not Erase” sign on the whiteboard and hope to come back and deal with it later. In many cases, they forget or the content is accidentally erased by other people. Furthermore, meeting participants who are on conference call from remote locations are not able to see the whiteboard content as the local participants do. In order to enable this, the 1 meeting sites often must be linked with expensive video conferencing equipments. Such equipment includes a pan-tilt-zoom camera which can be controlled by the remote participants. It is still not always satisfactory because of viewing angle, lighting variation, and image resolution, without mentioning lack of functionality of effective archiving and indexing of whiteboard contents. 

Our system was designed with three purposes: 
\begin{enumerate}
    \item Alleviate meeting participants the mundane tasks of note taking by capturing whiteboard content automatically or when the user asks;
    \item Communicate the whiteboard content to the remote meeting participants in real time using a fraction of the bandwidth required if video conferencing equipment is used;
    \item Archive the whole meeting in a way that a user (participants or not) can find efficiently the desired information. 
\end{enumerate}

To the best of our knowledge, all existing systems that capture whiteboard content in real time require instrumentation either in the pens or on the whiteboard. Our system allows the user to write freely on any existing whiteboard surface using any pen. To achieve this, our system uses an off the-shelf high-resolution video camera which captures images of the whiteboard at 7.5Hz. From the input video sequence, our algorithm separates people in the foreground from the whiteboard background and extracts the pen strokes as they are deposited to the whiteboard. To save bandwidth, only newly written pen strokes are compressed and sent to the remote participants.


\section{Existing solutions}
Currently there exist one solution developed by Zhengyon Zhang which is close by meaning to our project \cite{Zhengyou}. In Zhang’s paper was described about the main steps during the developing of the project: “The first step is then to localize the borders of the whiteboard in the image. This is done by detecting four strong edges. The whiteboard in an image usually appears to be a general quadrangle, rather than a rectangle, because of camera’s perspective projection. If a whiteboard does not have strong edges, an interface is provided for the user to specify the quadrangle manually. The second step is image rectification. For that, we first estimate the actual aspect ratio of the whiteboard from the quadrangle in the image based on the fact that it is a projection of a rectangle in space. From the estimated aspect ratio, and by choosing the “largest” whiteboard pixel as the standard pixel in the final image, we can compute the desired resolution of the final image. A planar mapping (a 3x3 homography matrix) is then computed from the original image quadrangle to the final image rectangle, and the whiteboard image is rectified accordingly. The last step is white balancing of the background color. This involves two procedures. The first is the estimation of the background color (the whiteboard color under the same lighting without anything written on it). This is not a trivial task because of complex lighting environment, whiteboard reflection and strokes written on the board. The second concerns the actual white balancing. We make the background uniformly white and increase color saturation of the pen strokes. The output is a crisp”\cite{Zhengyou}. 

During the development there were used good approach to divide the part of the image where person is located. Firstly, instead of analyzing the image pixel level, they divide each video frame into rectangular cell to reduce computational spent. Cell size is approximately the same as that which they expect the size of one character on the board (16 by 16 pixels at their implementation). Grid divides each frame in the input sequence of images in individual cells that are the basic unit in our analysis. Second, our analyzer structured as pipeline six analysis procedures. If the cell does not satisfy the images in a particular order, it will not be further processed by the following procedures in the pipeline. Thus, many cell images will not go through all six procedures. In the end, only a small number of cells containing the newly-born image pen strokes out of the analyzer.
One more advantage of their system was the presence of audio streaming. It means they save the audio process during the lecture.

To sum up, they have described the system board camera and projector system board cameras. The system board cameras, they have developed various techniques for computer vision improve performance in the use of physical whiteboard. In particular, the "board" Scan captures notes on the board, taking one or more images, thereby freeing participants. The meeting of the burden of manually copying the contents; "Real-Time board" allows users to share their ideas on the board in various scenarios, such as teleconferencing brainstorming sessions, lectures and meetings on project planning and patent disclosure, but only occupies part of it bandwidth and is suitable even for switched networks; "Whiteboard Backup" recording as board Activities and audio signals, and helps participants to effectively review the meeting in more recently, by providing key image frames that summarize the content of the board and structured Audio visual indexing. The system has been tested extensively, and have excellent results they were received.
They also identified the problem of visual echo cancellation in the projector-board cameras system and proposed a solution using both geometric calibration and color calibration. Visual Echo Cancellation is widely used in problems of working together in real time, either on site or remotely. The algorithm is tested on different layers and display content and good results. Both geometric calibration and color may be used for other purposes. Geometric Calibration Appliances were actually integrated into our camera-based projector human-computer interaction a system that tracks the position of the image of the laser spot to command the mouse cursor on the display screen. Some of the limitations of our current board camera projector systems are: 

\begin{enumerate}
\item Most of the surface of the board are not intended for projection display. Thus, they give more ref Basic than conventional screens for projectors. To avoid subsequent striking effect the audience, the projector should be placed on high (or very high or very low) Angle relative to the board.
\item We must repeat a color calibration, if some of the settings for the projector (color temperature, the contrast or brightness) or the camera (or the exposure, white balance) will be changed. Projecting and capturing $729 * n = 3645$ for $n = 5$ frames at 10 frames per second (for the design and Capture synchronized) takes about 6 minutes The second project is called “Rocket board”. This project appeared in kickstarter.com where you can present your project and ask for money to develop your project. But unfortunately they didn’t collect enough money to start their startup project.
\end{enumerate}