\chapter*{Conclusion}
\addcontentsline{toc}{chapter}{Conclusion}

Summing up, dissertation contains detection of board and transmission of the data between client and server in a compressed way. We presented a new approach of detection of whiteboard area. We used special ‘star’ pattern for turning any surface into whiteboard one. We also discussed about the structure of the star pattern in the geometric terms. The paper contains information about cascade training algorithms that are applied to the star pattern with a bunch of positive and negative samples. Moreover, we discussed and compared training algorithms against their learning time performance and false positive rates. Also we discussed about board detection by comparing well-known algorithms by using technique with four rectangles at the corners of the board. This four rectangles says how large the board will be. The new approach of using four rectangles results were better than any other edge detection algorithms. The edge detection algorithms are Canny, Prewitt, Sobel, Laplacian etc. In this dissertation paper was described the comparison between Canny and our new approach of board detection. New approach was better on its number of candidates for the whiteboard corners. This new method has disadvantages too. It needs two frames to compare between each other to take differences. This two frames will be taken from before and after the rectangles will be drawn on the whiteboard. After detecting all four rectangles there was used morphological operators on an image. After that we will have four connected components. Then was made bounding box and the each box is validated for rectangle shape. Also all of the candidates were validated through their perimeters and diagonals.

The second research was done on compression of the data which needs to be sent to the server for further share at the website to make it available for everyone even with low Internet connection. The research in this field was began from sending to the server each frame which was cropped as a board by four corners. However each image took approximately 1.5MB memory it was traffic expensive each time to send to server. The next step was to send to the server a string of all the changed pixels. It worked much faster than sending each image to the server. While generating a string which contains all this points was a little bit expensive and need to be improved. At the next step there was used to compress all this points into one object like rectangle. If the shape of all points which are needed to send to the server they are more close to the rectangle. It is obvious that there are more horizontal and vertical lines at the whiteboard rather than curves and ovals etc. So it was came to decision to describe all this points as a rectangle. It means this there will be only the beginning x and y points of the collection of points which is going to be observed and width and height of the rectangle to say where it is going to finish. By using this greedy algorithm the memory to describe was decreased from 13MB till 1.4MB which is approximately 10-15 times. Also this rectangles are easier to draw in the server side too. Because it is not need to draw each pixel in the canvas. It is enough to show beginning and end of the rectangle and then the javascript efficiently draws the required points on the canvas.

The next step in this research was to build archive of the points to make it possible to replay any time. This idea suggests to remove all the video and save all of them only in the way of archive of strings. It will decrease the spent memory on the server and decrease the traffic exchange between client and server. The process of archiving is done at the stage of sending all this compressed points to the server. Each time when the server gets the data from the client it saves it to the database by its session id including the time. However we have all needed points with its time when it need to be drawn it becomes easy to make an archive of video with ability of further replay. 

In perspectives of the application is to make the detection of board by using the machine learning. It will be more flexible in detecting the board. Another futures of the application may the shortcuts. For example recognition of patterns on the board. If there is drawn sun on the board in the database it is saved as an image of the sun and when presenting to the user the server will take the picture of the sun from its memory and draw a beautiful image of sun. Exactly this tool may by very interesting for school children. Also there may build library for each specific fields. For example programmers can use shortcuts like highlighting the code which written on the board and etc. 

This application can be used widely not only for study purpose it can be used also for brainstorming, board conferencing and etc.   
