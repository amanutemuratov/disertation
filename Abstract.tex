\begin{abstract}
\hspace{1cm}Master dissertation consists of \pageref{LastPage} pages of the typewritten text, 30 figures, 2 tables and 31 reference materials. The following keywords and terms are used in the paper:

CAMERA, WHITEBOARD, COMPRESSION, CLEAN, SERVER, ARCHIVE, OPENCV, EDGE, DETECTION, CONTOUR DETECTION, ALGORITHM, E-LEARNING

Dissertation contains an overview of an E-learning advantages in a modern world. Also it contains different research analysis and techniques for whiteboard detection and image processing.

Today e-learning is gaining more and more turns and each year requires more memory to store data. Among stored data video occupies more memory than others. The main goal is to create a system which will track the changes on the whiteboard and send all the data to server in a compressed form. At server side is to get all data and build archive which can be restored. The main idea is to replace video with archive which occupies less memory.

	
\end{abstract}

\begin{otherlanguage}{russian}
\begin{abstract}
\hspace{1cm}Магистерская диссертация содежрит \pageref{LastPage} страниц, 30 иллюстрации, 2 таблицы, список использованных источников - 31 наименований. Использованы следующие ключевые слова:

КАМЕРА, ДОСКА, СЖАТИЕ, КЛИЕНТ, СЕРВЕР, АРХИВ, OPENCV, ОБНАРУЖЕНИЕ КРАЯ, ОБНАРУЖЕНИЕ КОНТУРА, АЛГОРИТМ

Диссертация содержит обзор преимуществ электронного обучения в современном мире. Также содержит различные анализы и методы для обнаружения доски и обработки изображений.
	
На сегодняшний день онлайн обучение набирает все больше оборотов и с каждым годом требует все больше памяти для хранения данных. Среди них видео занимает самую большую память. Основная цель заключается в создании системы, которая будет отслеживать изменения на доске и передавать все данные на сервер в сжатом виде, а на стороны сервера, получить все данные и построить архив, который может быть восстановлен. Основная идея заключается в том, чтобы заменить видео на архив, который занимает меньше памяти.


\end{abstract}
\end{otherlanguage}

\begin{otherlanguage}{russian}
\begin{abstract}
\hspace{1cm}Диссертациялық жұмыс машинаға басылған мәтіннің \pageref{LastPage} бетінен, 30 кестеден, 2 суреттен және 31 қолданылған әдебиеттер санынан тұрады. Келесi кiлт сөздердi пайдаланыңыз:

КАМЕРА, ТАҚТА, СЫҒУ, КЛИЕНТ, СЕРВЕР, МҰРАҒАТ, OPENCV, ҚЫРДЫ АНЫҚТАУ, БАҒДАРЛАМА

Диссертация қазіргі таңдағы электрондық оқыту артықшылықтары жөніндегі ақпараттарды қамтиды. Сондай-ақ, ол тақтаны анықтауға және сурет өңдеуге арналған әр түрлі ғылыми зерттеу, талдау әдістерін қамтиды.

Қазіргі таңда элетронды түрде білім алу қарқынды даму үстінде. Сақтау керек мағлұматтардың жады жыл санап артуда. Соның ішінде ең көп жадты қажет ететін видео болып табылады. Негізгі мақсат тақтада болып жатқан өзгерістерді сығылған турде серверге жіберіп, онда қайта қалпына келтіру мүмкіндігімен видеоның мұрағатын құру болып табылады. Видеоны мұрағатқа алмастыру негізгі ой болып табылады

\end{abstract}
\end{otherlanguage}
