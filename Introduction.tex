\chapter{Introduction}

%\addcontentsline{toc}{chapter}{Introduction}
Education is one of the most important tools to succeed in our life. Development of economy brought to rapid development of information technology. In twenty first century it is difficult to imagine our life without new technologies and gadgets. All of this brought to be in trend with latest tools and technologies. However the foundation of any company is their staff. So, each company which wants to see own selves at the pick of the pedestal need theirs employees to be equipped with latest information and advanced skills. \cite{Edith} To fill this gaps there are provided a lot of online courses where any people can gain knowledge in a particular subject. . Nowadays there are a lot of online courses with multimedia, pictures, illustration materials which makes easy for understanding the topic. The advantages of E-learning are:
\begin{itemize}
\item Learner centered
\item Time and location flexible
\item Cost-effective for learners
\item Potentially available to global audience
\item Unlimited access to knowledge
\item Archival capability for knowledge reuse and sharing \cite{Dansong}
\end{itemize}
The system of automatically detecting digital images during the lecture would be a great contribution to the area of e-learning. In this paper, we describe a system that takes lecture images from a whiteboard that cropped in real time and send to the server to create a presentation. The application incorporates a variety of automatic and interactive techniques to identify and segment desired content in the camera view, allowing the user to publish a more focused video. There are can be several images on the whiteboard. Required region will be detected by labels in the form of little squares or star patterns. Cropped images will be sent to the server, and presentation creates.
The system consists from several stages: detecting the board, optimization of data transmission between client and server, image processing. Detecting the board starts with determining rectangle as a board. System takes image frames from the video stream. Streaming is provided via camera which is directed to the region of interest. ROI has great practical importance in computer vision operations to detect subregion of the image. The ROI based approach is more accurate in describing the image content than using global features [8]. System will identify the corners of the board by detecting the rectangles which will be drawn by the lecturer or a user. Before and after the process of drawing rectangles the system will use the motion detection algorithms for detecting the hand cover that will separate two images of before the motion and after the motion. Collins et al. \cite{Collins} described a different hybrid algorithm for motion detection. Assuming that the background will be static, a three-frame differencing operation that takes frames at time $t$, $t - 1$ and $t - 2$ performing the differentiating and multiplication operations to determine regions of legitimate motion and to erase ghosting.
Having the determined corners of the whiteboard image, as the next part of the research we used transformation and rotation of matrix into normal form. It needs to convert the images that shoot by some angle less than 90 deg. There is used algorithm for automatic perspective correction for quadrilateral objects \cite{Pooja}. Afterwards in each step of processing the image there can be a several types of noise that can block the actual information which should be identified. The noises appearing as a human body is determined using algorithms for contour detection \cite{Ruchika}. The noise of shadows from objects is also considered with different filtering algorithms \cite{Lin}. 
After getting clear image of the whiteboard without noise, we can send the information about image to the server. There are several ways to perform the transmission of data to the server: sending the full image, sending each changed pixels, sending changed pixels as a combination of line-segments. Sending the full image process is the easiest way for data transmission between client and server, because of the support of advanced libraries by the modern high-level programming languages. Nevertheless, it is the most expensive one in case of traffic load. \cite{Jayanta} Rather than sending the full information about a board, the effective way is to send only changed pixels. This method dramatically increases the traffic load efficiency. Combining the changed pixels into line-segments can optimize the data furthermore. Each time server will get an information, the state of changes in the whiteboard. Then it will immediately decode the received message and draw corresponding surface. While drawing, system can identify some patterns that matches with a predefined dataset of knowledge base of different shortcuts. For each identified shortcut it will produce a corresponding result. In the process of identification there is used pattern recognition algorithms and machine learning techniques \cite{Dansong}.
We presented a new approach to detect of digital images under the assumption that either the camera that took the image or sufficiently many images taken by that camera are available. Methods automatically determine the required area without assuming any a prior knowledge.
Our system can be used by lecturers on the universities, schools, e-learning, and even brainstorming.
